\documentclass{resume} % Use the custom resume.cls style

\usepackage[left=0.4 in,top=0.4in,right=0.4 in,bottom=0.4in]{geometry} % Document margins
\usepackage{kotex}
\newcommand{\tab}[1]{\hspace{.2667\textwidth}\rlap{#1}} 
\newcommand{\itab}[1]{\hspace{0em}\rlap{#1}}
\name{GWANHUI LEE} % Your name
% You can merge both of these into a single line, if you do not have a website.
\address{+82(10) 2376-5234 \\ ULSAN, KOREA} 
\address{\href{mailto:rhdqn5234@yonsei.ac.kr}{rhdqn5234@yonsei.ac.kr} \\ \href{https://github.com/LeeGwanHui}{github.com/LeeGwanHui}}  %

\begin{document}

%----------------------------------------------------------------------------------------
%----------------------------------------------------------------------------------------
%	EDUCATION SECTION
%----------------------------------------------------------------------------------------

\begin{rSection}{Education}

{\bf 울산삼일초등학교} \hfill {2005~2011}\\
{\bf 울산중학교} \hfill {2011~2014}\\
{\bf 울산과학고등학교} \hfill {2014~2017}\\
{\bf Yonsei University}, SEOUL \hfill {2017~~~~~}\\
{\bf Republic of Korea Air Force} \hfill {2018~2019}\\

\end{rSection}

%----------------------------------------------------------------------------------------
% TECHINICAL STRENGTHS	
%----------------------------------------------------------------------------------------
\begin{rSection}{SKILLS}
\begin{tabular}{ @{} >{\bfseries}l @{\hspace{6ex}} l }
Languages & Python, kotlin, java, C, SQL \\
Frameworks & Pytorch \\
Tools & Git, Latex, pycharm, unity, android studio \\
\\
\end{tabular}\\
\end{rSection}

%----------------------------------------------------------------------------------------
% Projects
%----------------------------------------------------------------------------------------
\begin{rSection}{PROJECTS}
\vspace{-1.25em}
\item \textbf{Project 1} {Kotlin, Android Studio} \hfill \href{https://github.com/LeeGwanHui/Rock-Scissors-paper-kotlin}{GitHub}
\begin{itemize}
    \itemsep -3pt {} 
     \item 이 프로젝트는 가위바위보 하나빼기를 구현한 프로그램입니다. 
     \item 총 3명의 학생이 참여했으며 각각 가위바위보를 classification과 서버구축, Unity를 이용한 rending, 스마트폰의 UI와 서버와의 통신을 담당했습니다.
    \item 위의 역할 중 제가 맡았던 역할은 스마트폰의 UI와 서버와의 통신을 담당했습니다.
 \end{itemize}
\item \textbf{Project 2} {python, pytorch, unet, segmentation} \hfill \href{https://github.com/LeeGwanHui/lumbar}{GitHub}
\begin{itemize}
    \itemsep -3pt {} 
     \item unet을 이용해서 lumbar+ spine segmentation
     \item 최종 결과 연세대학교 황도식 교수님 기초인공지능 수업 챌린지 대쉬보드 4등
    \item 데이터 파일만 주고 모든 것을 자유롭게 코딩하여 결과를 내는 것이 목표 
    \item 데이터 파일 형태는 dcm(DICOM file : 의료쪽에서 사용하는 CT, MRI 등의 file 형태), mat(label file).
 \end{itemize}
\item \textbf{Project 3} {SQL, comento 직무 특강 SQL 입문부터 활용까지 - 데이터 분석 보고서 작성과 대시보드 개발 } \hfill \href{https://drive.google.com/file/d/1PEf-mlerQgi70gw5mVsfADWKnZp_6UBb/view?usp=sharing}{수료증}
\begin{itemize}
    \itemsep -3pt {} 
     \item SQL을 이용해서 Northwind Database(가상의 식품 회사에 대한 데이터 베이스)를 분석.
     \item 
 \end{itemize}
\end{rSection} 

%----------------------------------------------------------------------------------------
\begin{rSection}{Extra-Curricular Activities or certificate} 
\begin{itemize}
    \item 	정보처리 산업기사 , 지게차 기능사
    \item	어학 성적(영어를 못함) 토익: 725점 토익스피킹: level 5
    \item   전공기초 과목(데이터 구조, 공학수학) 튜터링 2회 (전기전자공학부 명예학회 활동)
\end{itemize}


\end{rSection}

%------------------------
% Use this more detailed section if you have Relevant work experience
% keep your resume to 1 page, if you need to remove a project to display relevant experience
% that is okay
% ----------------------------
% \begin{rSection}{EXPERIENCE}

% \textbf{Role Name} \hfill Jan 2017 - Jan 2019\\
% Company Name \hfill \textit{San Francisco, CA}
%  \begin{itemize}
%     \itemsep -3pt {} 
%      \item Achieved X\% growth for XYZ using A, B, and C skills.
%      \item Led XYZ which led to X\% of improvement in ABC
%     \item Developed XYZ that did A, B, and C using X, Y, and Z. 
%  \end{itemize}
 
% \textbf{Role Name} \hfill Jan 2017 - Jan 2019\\
% Company Name \hfill \textit{San Francisco, CA}
%  \begin{itemize}
%     \itemsep -3pt {} 
%      \item Achieved X\% growth for XYZ using A, B, and C skills.
%      \item Led XYZ which led to X\% of improvement in ABC
%     \item Developed XYZ that did A, B, and C using X, Y, and Z. 
%  \end{itemize}

% \end{rSection} 
%----------------------------------------------------------------------------------------



\end{document}
